\documentclass[DM,lsstdraft,STR,toc]{lsstdoc}
\usepackage{geometry}
\usepackage{longtable,booktabs}
\usepackage{enumitem}
\usepackage{arydshln}

\input meta.tex

\providecommand{\tightlist}{
  \setlength{\itemsep}{0pt}\setlength{\parskip}{0pt}}

\setcounter{tocdepth}{4}

\begin{document}

\def\milestoneName{RSP redeployed on the Interim Data Facility (IDF), ready for DP0.1}
\def\milestoneId{LDM-503-14a:}
\def\product{LSP Services}

\setDocCompact{true}

\title{ LDM-503-14a: RSP redeployed on the Interim Data Facility (IDF), ready for DP0.1 Test Plan and Report}
\setDocRef{\lsstDocType-\lsstDocNum}
\date{\vcsdate}
\setDocUpstreamLocation{\url{https://github.com/lsst/lsst-texmf/examples}}
\author{ Gregory Dubois-Felsmann }

\input history_and_info.tex


\setDocAbstract{
This is the test plan and report for LDM-503-14a: (RSP redeployed on the Interim Data Facility (IDF), ready for DP0.1),
an LSST level 2 milestone pertaining to the Data Management Subsystem.
}


\maketitle

\section{Introduction}
\label{sect:intro}


\subsection{Objectives}
\label{sect:objectives}

 Demonstrate that the end-of-FY2020 capabilities of the Rubin Science
Platform have been made available on the Interim Data Facility, and that
DP0.1, based on ingested externally-provided DC2 data, can be supported.
~May be demonstrated with the DC2 DP0.1 dataset itself or with a dataset
of equivalent complexity.\\[2\baselineskip]DP0.1 expectations are as
described in \href{https://rtn-001.lsst.io/}{RTN-001} and
\href{https://rtn-004.lsst.io/}{RTN-004} .



\subsection{System Overview}
\label{sect:systemoverview}



\subsection{Document Overview}
\label{sect:docoverview}

This document was generated from Jira, obtaining the relevant information from the 
\href{https://jira.lsstcorp.org/secure/Tests.jspa#/testPlan/LVV-P79}{LVV-P79}
~Jira Test Plan and related Test Cycles (
  \href{https://jira.lsstcorp.org/secure/Tests.jspa#/testCycle/LVV-C166}{LVV-C166}
).

Section \ref{sect:intro} provides an overview of the test campaign, the system under test (\product{}),
the applicable documentation, and explains how this document is organized.
Section \ref{sect:configuration}  describes the configuration used for this test.
Section \ref{sect:personnel} describes the necessary roles and lists the individuals assigned to them.
%Section \ref{sect:plannedtestactivities} provides the list of planned test cycles and test cases,
including all relevant information that fully describes the test campaign.

Section \ref{sect:overview} provides a summary of the test results, including an overview in Table \ref{table:summary},
an overall assessment statement and suggestions for possible improvements.
Section \ref{sect:detailedtestresults} provides detailed results for each step in each test case.

The current status of test plan LVV-P79 in Jira is \textbf{ Draft }.

\subsection{References}
\label{sect:references}
\renewcommand{\refname}{}
\bibliography{lsst,refs,books,refs_ads}
\section{Test Configuration}
\label{sect:configuration}

\subsection{Data Collection}

  Observing is not required for this test campaign.

\subsection{Verification Environment}
\label{sect:hwconf}
  Must be executed in a well-documented controlled state of the IDF.




\newpage
\section{Personnel}
\label{sect:personnel}

The personnel involved in the test campaign are shown in the following table.

\begin{longtable}{p{3cm}p{3cm}p{3cm}p{6cm}}
\hline
\multicolumn{2}{r}{Test Plan (LVV-P79) owner:} &
\multicolumn{2}{l}{\textbf{ Gregory Dubois-Felsmann } }\\\hline
\multicolumn{2}{r}{ LVV-C166 owner:} &
\multicolumn{2}{l}{\textbf{
    Gregory Dubois-Felsmann
}
} \\\hline
\textbf{Test Case} & \textbf{Assigned to} & \textbf{Executed by} & \textbf{Additional Test Personnel} \\ \hline
\href{https://jira.lsstcorp.org/secure/Tests.jspa#/testCase/LVV-T2171}{LVV-T2171}
& {\small Gregory Dubois-Felsmann } & {\small  } &
\begin{minipage}[]{6cm}
\smallskip
{\small Someone with credentials allowing access to the instance of the RSP at
the IDF on which the data are deployed. }
\medskip
\end{minipage}
\\ \hline
\href{https://jira.lsstcorp.org/secure/Tests.jspa#/testCase/LVV-T2172}{LVV-T2172}
& {\small Gregory Dubois-Felsmann } & {\small  } &
\begin{minipage}[]{6cm}
\smallskip
{\small  }
\medskip
\end{minipage}
\\ \hline
\end{longtable}

\newpage

\section{Test Campaign Overview}
\label{sect:overview}

\subsection{Summary}
\label{sect:summarytable}

\begin{longtable}{p{2cm}p{2.5cm}p{9cm}p{2.5cm}}
\toprule
\multicolumn{3}{p{13.5cm}}{ Test Plan {\bf LVV-P79: LDM-503-14a: RSP redeployed on the Interim Data Facility (IDF), ready
for DP0.1 }} & Draft \\\hline

  \multicolumn{3}{p{13.5cm}}{ Test Cycle {\bf LVV-C166: milestone-dp01: Test RSP capabilities on IDF for DP0.1 readiness }} & Not Executed \\\hline

  {\bf \footnotesize test case} & {\bf \footnotesize status} & {\bf \footnotesize comment} & {\bf \footnotesize issues} \\\toprule

\href{https://jira.lsstcorp.org/secure/Tests.jspa#/testCase/LVV-T2171}{LVV-T2171}
    & Not Executed &
    \begin{minipage}[]{9cm}
    \smallskip
    
    \medskip
    \end{minipage}
    &
    \\\hline
\href{https://jira.lsstcorp.org/secure/Tests.jspa#/testCase/LVV-T2172}{LVV-T2172}
    & Not Executed &
    \begin{minipage}[]{9cm}
    \smallskip
    
    \medskip
    \end{minipage}
    &
    \\\hline
\caption{Test Campaign Summary}
\label{table:summary}
\end{longtable}

\subsection{Overall Assessment}
\label{sect:overallassessment}

Not yet available.

\subsection{Recommended Improvements}
\label{sect:recommendations}

Not yet available.

\newpage
\section{Detailed Test Results}
\label{sect:detailedtestresults}

\subsection{Test Cycle LVV-C166 }

Open test cycle {\it \href{https://jira.lsstcorp.org/secure/Tests.jspa#/testrun/LVV-C166}{milestone-dp01: Test RSP capabilities on IDF for DP0.1 readiness}} in Jira.

milestone-dp01: Test RSP capabilities on IDF for DP0.1 readiness\\
Status: Not Executed

This test cycle contains the tests necessary to verify the readiness of
the RSP as redeployed on the IDF to meet the needs of the DP0.1
exercise, essentially repeating tests previously carried out on the NCSA
RSP deployments.

\subsubsection{Software Version/Baseline}
Not provided.

\subsubsection{Configuration}
Not provided.

\subsubsection{Test Cases in LVV-C166 Test Cycle}

\paragraph{Test Case LVV-T2171 - Notebook Aspect access to a DP0.1 dataset in the IDF-deployed RSP }\mbox{}\\

Open  \href{https://jira.lsstcorp.org/secure/Tests.jspa#/testCase/LVV-T2171}{\textit{ LVV-T2171 } }
test case in Jira.

Verify the availability through the Notebook Aspect of the DP0.1 test
dataset or an equivalent, including access to both catalogs and images
via the Butler.\\[2\baselineskip]

\textbf{ Preconditions}:\\
Creation of the DP0.1 dataset or a stand-in, in the form of a Butler
repository accessible from the Notebook Aspect and with associated
catalog data in a TAP service in the same RSP instance at the IDF.

Execution status: {\bf Not Executed }

Final comment:\\


Detailed steps results:

\begin{longtable}{p{1cm}p{15cm}}
\hline
{Step} & Step Details\\ \hline
1 & Description \\
 & \begin{minipage}[t]{15cm}
{\footnotesize
Authenticate to the notebook aspect of the LSST Science Platform
(NB-LSP). ~This is currently at
https://lsst-lsp-stable.ncsa.illinois.edu/nb.

\medskip }
\end{minipage}
\\ \cdashline{2-2}


 & Expected Result \\
 & \begin{minipage}[t]{15cm}{\footnotesize
Redirection to the spawner page of the NB-LSP allowing selection of the
containerized stack version and machine flavor.

\medskip }
\end{minipage} \\ \cdashline{2-2}

 & Actual Result \\
 & \begin{minipage}[t]{15cm}{\footnotesize

\medskip }
\end{minipage} \\ \cdashline{2-2}

 & Status: \textbf{ Not Executed } \\ \hline

2 & Description \\
 & \begin{minipage}[t]{15cm}
{\footnotesize
Spawn a container by:\\
1) choosing an appropriate stack version: e.g. the latest weekly.\\
2) choosing an appropriate machine flavor: e.g. medium\\
3) click ``Spawn''

\medskip }
\end{minipage}
\\ \cdashline{2-2}


 & Expected Result \\
 & \begin{minipage}[t]{15cm}{\footnotesize
Redirection to the JupyterLab environment served from the chosen
container containing the correct stack version.

\medskip }
\end{minipage} \\ \cdashline{2-2}

 & Actual Result \\
 & \begin{minipage}[t]{15cm}{\footnotesize

\medskip }
\end{minipage} \\ \cdashline{2-2}

 & Status: \textbf{ Not Executed } \\ \hline

3 & Description \\
 & \begin{minipage}[t]{15cm}
{\footnotesize
Use the file browser on the left of the JupyterLab UI to open the ``LSST
Catalog Access Tutorial'' notebook.

\medskip }
\end{minipage}
\\ \cdashline{2-2}


 & Expected Result \\
 & \begin{minipage}[t]{15cm}{\footnotesize

\medskip }
\end{minipage} \\ \cdashline{2-2}

 & Actual Result \\
 & \begin{minipage}[t]{15cm}{\footnotesize

\medskip }
\end{minipage} \\ \cdashline{2-2}

 & Status: \textbf{ Not Executed } \\ \hline

4 & Description \\
 & \begin{minipage}[t]{15cm}
{\footnotesize
Execute the notebook. ~Take note of any errors encountered along the
way.

\medskip }
\end{minipage}
\\ \cdashline{2-2}


 & Expected Result \\
 & \begin{minipage}[t]{15cm}{\footnotesize

\medskip }
\end{minipage} \\ \cdashline{2-2}

 & Actual Result \\
 & \begin{minipage}[t]{15cm}{\footnotesize

\medskip }
\end{minipage} \\ \cdashline{2-2}

 & Status: \textbf{ Not Executed } \\ \hline

5 & Description \\
 & \begin{minipage}[t]{15cm}
{\footnotesize
Use the file browser on the left of the JupyterLab UI to open the
``Firefly'' notebook.

\medskip }
\end{minipage}
\\ \cdashline{2-2}


 & Expected Result \\
 & \begin{minipage}[t]{15cm}{\footnotesize

\medskip }
\end{minipage} \\ \cdashline{2-2}

 & Actual Result \\
 & \begin{minipage}[t]{15cm}{\footnotesize

\medskip }
\end{minipage} \\ \cdashline{2-2}

 & Status: \textbf{ Not Executed } \\ \hline

6 & Description \\
 & \begin{minipage}[t]{15cm}
{\footnotesize
Execute the notebook. ~Take note of any errors encountered along the
way.

\medskip }
\end{minipage}
\\ \cdashline{2-2}


 & Expected Result \\
 & \begin{minipage}[t]{15cm}{\footnotesize

\medskip }
\end{minipage} \\ \cdashline{2-2}

 & Actual Result \\
 & \begin{minipage}[t]{15cm}{\footnotesize

\medskip }
\end{minipage} \\ \cdashline{2-2}

 & Status: \textbf{ Not Executed } \\ \hline

7 & Description \\
 & \begin{minipage}[t]{15cm}
{\footnotesize
Log out of the Notebook Aspect.

\medskip }
\end{minipage}
\\ \cdashline{2-2}


 & Expected Result \\
 & \begin{minipage}[t]{15cm}{\footnotesize

\medskip }
\end{minipage} \\ \cdashline{2-2}

 & Actual Result \\
 & \begin{minipage}[t]{15cm}{\footnotesize

\medskip }
\end{minipage} \\ \cdashline{2-2}

 & Status: \textbf{ Not Executed } \\ \hline

\end{longtable}

\paragraph{Test Case LVV-T2172 - Portal Aspect access to a DP0.1 dataset in the IDF-deployed RSP }\mbox{}\\

Open  \href{https://jira.lsstcorp.org/secure/Tests.jspa#/testCase/LVV-T2172}{\textit{ LVV-T2172 } }
test case in Jira.

Verify the availability through the Portal Aspect to catalog data from
the DP0.1 test dataset or an equivalent, via an RSP TAP service on the
IDF. ~The emphasis will be on an Object-like catalog.\\[2\baselineskip]

\textbf{ Preconditions}:\\
Creation of the DP0.1 dataset or a stand-in, and service of the
associated catalog data and schema in a TAP service in the same RSP
instance at the IDF.

Execution status: {\bf Not Executed }

Final comment:\\


Detailed steps results:

\begin{longtable}{p{1cm}p{15cm}}
\hline
{Step} & Step Details\\ \hline
1 & Description \\
 & \begin{minipage}[t]{15cm}
{\footnotesize
Navigate to the Portal Aspect endpoint. ~The stable version should be
used for this test and is currently located at:
https://lsst-lsp-stable.ncsa.illinois.edu/portal/app/ .

\medskip }
\end{minipage}
\\ \cdashline{2-2}


 & Expected Result \\
 & \begin{minipage}[t]{15cm}{\footnotesize
A credential-entry screen should be displayed.

\medskip }
\end{minipage} \\ \cdashline{2-2}

 & Actual Result \\
 & \begin{minipage}[t]{15cm}{\footnotesize

\medskip }
\end{minipage} \\ \cdashline{2-2}

 & Status: \textbf{ Not Executed } \\ \hline

2 & Description \\
 & \begin{minipage}[t]{15cm}
{\footnotesize
Enter a valid set of credentials for an LSST user with LSP access on the
instance under test.

\medskip }
\end{minipage}
\\ \cdashline{2-2}


 & Expected Result \\
 & \begin{minipage}[t]{15cm}{\footnotesize
The Portal Aspect UI should be displayed following authentication.

\medskip }
\end{minipage} \\ \cdashline{2-2}

 & Actual Result \\
 & \begin{minipage}[t]{15cm}{\footnotesize

\medskip }
\end{minipage} \\ \cdashline{2-2}

 & Status: \textbf{ Not Executed } \\ \hline

3 & Description \\
 & \begin{minipage}[t]{15cm}
{\footnotesize
Navigate to the TAP Search screen

\medskip }
\end{minipage}
\\ \cdashline{2-2}


 & Expected Result \\
 & \begin{minipage}[t]{15cm}{\footnotesize

\medskip }
\end{minipage} \\ \cdashline{2-2}

 & Actual Result \\
 & \begin{minipage}[t]{15cm}{\footnotesize

\medskip }
\end{minipage} \\ \cdashline{2-2}

 & Status: \textbf{ Not Executed } \\ \hline

4 & Description \\
 & \begin{minipage}[t]{15cm}
{\footnotesize
Ensure that the TAP service internal to the RSP instance is selected.
~(This should be the default choice.)

\medskip }
\end{minipage}
\\ \cdashline{2-2}


 & Expected Result \\
 & \begin{minipage}[t]{15cm}{\footnotesize
A list of ``schemas'' available on that service should be displayed,
along with a list of tables in the default schema.

\medskip }
\end{minipage} \\ \cdashline{2-2}

 & Actual Result \\
 & \begin{minipage}[t]{15cm}{\footnotesize

\medskip }
\end{minipage} \\ \cdashline{2-2}

 & Status: \textbf{ Not Executed } \\ \hline

5 & Description \\
 & \begin{minipage}[t]{15cm}
{\footnotesize
Select the TAP ``schema'' for the data to be queried (see test
parameter). ~

\medskip }
\end{minipage}
\\ \cdashline{2-2}

 & Test Data \\
 & \begin{minipage}[t]{15cm}{\footnotesize
{\{TAP ``schema'' name for catalog data to be tested\}}⁠~

\medskip }
\end{minipage} \\ \cdashline{2-2}

 & Expected Result \\
 & \begin{minipage}[t]{15cm}{\footnotesize
A list of tables in the selected schema should be displayed.

\medskip }
\end{minipage} \\ \cdashline{2-2}

 & Actual Result \\
 & \begin{minipage}[t]{15cm}{\footnotesize

\medskip }
\end{minipage} \\ \cdashline{2-2}

 & Status: \textbf{ Not Executed } \\ \hline

6 & Description \\
 & \begin{minipage}[t]{15cm}
{\footnotesize
Select the catalog table to be queried (see test parameter). ~

\medskip }
\end{minipage}
\\ \cdashline{2-2}

 & Test Data \\
 & \begin{minipage}[t]{15cm}{\footnotesize
{\{TAP ``table'' name for catalog to be queried\}}⁠~

\medskip }
\end{minipage} \\ \cdashline{2-2}

 & Expected Result \\
 & \begin{minipage}[t]{15cm}{\footnotesize
A search interface for the selected table should be presented.

\medskip }
\end{minipage} \\ \cdashline{2-2}

 & Actual Result \\
 & \begin{minipage}[t]{15cm}{\footnotesize

\medskip }
\end{minipage} \\ \cdashline{2-2}

 & Status: \textbf{ Not Executed } \\ \hline

7 & Description \\
 & \begin{minipage}[t]{15cm}
{\footnotesize
Enter the sky coordinates of the location to be tested (see test
parameter) in the ``Spatial'' query-builder element on the left of the
screen. ~(Note that the test dataset is likely to be of limited extent
on the sky.) Enter 100 arcseconds as the search radius.

\medskip }
\end{minipage}
\\ \cdashline{2-2}

 & Test Data \\
 & \begin{minipage}[t]{15cm}{\footnotesize
{\{(ra, dec) tuple for a location covered by the test catalog\}}⁠~

\medskip }
\end{minipage} \\ \cdashline{2-2}

 & Expected Result \\
 & \begin{minipage}[t]{15cm}{\footnotesize

\medskip }
\end{minipage} \\ \cdashline{2-2}

 & Actual Result \\
 & \begin{minipage}[t]{15cm}{\footnotesize

\medskip }
\end{minipage} \\ \cdashline{2-2}

 & Status: \textbf{ Not Executed } \\ \hline

8 & Description \\
 & \begin{minipage}[t]{15cm}
{\footnotesize
Verify that a list of available columns is displayed on the right of the
search screen. ~Note in the test report whether a subset of the
available columns is highlighted with a check mark, and if so which
columns they are.

\medskip }
\end{minipage}
\\ \cdashline{2-2}


 & Expected Result \\
 & \begin{minipage}[t]{15cm}{\footnotesize

\medskip }
\end{minipage} \\ \cdashline{2-2}

 & Actual Result \\
 & \begin{minipage}[t]{15cm}{\footnotesize

\medskip }
\end{minipage} \\ \cdashline{2-2}

 & Status: \textbf{ Not Executed } \\ \hline

9 & Description \\
 & \begin{minipage}[t]{15cm}
{\footnotesize
Execute the search.

\medskip }
\end{minipage}
\\ \cdashline{2-2}


 & Expected Result \\
 & \begin{minipage}[t]{15cm}{\footnotesize
Possibly after the display of an in-progress indication, a search result
should be displayed in the ``tri-view'' - a coverage image on the upper
left, a default X-Y plot on the upper right, and the tabular query
result on the bottom.\\[2\baselineskip]Note that if the dataset is
simulated, the coverage image may not correspond to the catalog data.
~(It is not a requirement of DP0.1 for a coverage image for DESC DC2 to
be created or made available in the Portal Aspect.)

\medskip }
\end{minipage} \\ \cdashline{2-2}

 & Actual Result \\
 & \begin{minipage}[t]{15cm}{\footnotesize

\medskip }
\end{minipage} \\ \cdashline{2-2}

 & Status: \textbf{ Not Executed } \\ \hline

10 & Description \\
 & \begin{minipage}[t]{15cm}
{\footnotesize
Verify that the query result covers the expected region of sky, and that
the expected set of columns is included in the query result. ~Record the
number of rows returned by the query.\\[2\baselineskip]Record the
identity displayed for the coverage image.

\medskip }
\end{minipage}
\\ \cdashline{2-2}


 & Expected Result \\
 & \begin{minipage}[t]{15cm}{\footnotesize

\medskip }
\end{minipage} \\ \cdashline{2-2}

 & Actual Result \\
 & \begin{minipage}[t]{15cm}{\footnotesize

\medskip }
\end{minipage} \\ \cdashline{2-2}

 & Status: \textbf{ Not Executed } \\ \hline

11 & Description \\
 & \begin{minipage}[t]{15cm}
{\footnotesize
Verify that the X-Y plot can be modified to display a user-selected pair
of columns.

\medskip }
\end{minipage}
\\ \cdashline{2-2}


 & Expected Result \\
 & \begin{minipage}[t]{15cm}{\footnotesize

\medskip }
\end{minipage} \\ \cdashline{2-2}

 & Actual Result \\
 & \begin{minipage}[t]{15cm}{\footnotesize

\medskip }
\end{minipage} \\ \cdashline{2-2}

 & Status: \textbf{ Not Executed } \\ \hline

12 & Description \\
 & \begin{minipage}[t]{15cm}
{\footnotesize
Verify that highlighted rows in the table, points in the X-Y plot, and
marks on the coverage image are connected.

\medskip }
\end{minipage}
\\ \cdashline{2-2}


 & Expected Result \\
 & \begin{minipage}[t]{15cm}{\footnotesize

\medskip }
\end{minipage} \\ \cdashline{2-2}

 & Actual Result \\
 & \begin{minipage}[t]{15cm}{\footnotesize

\medskip }
\end{minipage} \\ \cdashline{2-2}

 & Status: \textbf{ Not Executed } \\ \hline

13 & Description \\
 & \begin{minipage}[t]{15cm}
{\footnotesize
Verify that selections made in the three panes of the tri-view are
reflected in the other panes.

\medskip }
\end{minipage}
\\ \cdashline{2-2}


 & Expected Result \\
 & \begin{minipage}[t]{15cm}{\footnotesize

\medskip }
\end{minipage} \\ \cdashline{2-2}

 & Actual Result \\
 & \begin{minipage}[t]{15cm}{\footnotesize

\medskip }
\end{minipage} \\ \cdashline{2-2}

 & Status: \textbf{ Not Executed } \\ \hline

14 & Description \\
 & \begin{minipage}[t]{15cm}
{\footnotesize
Use the ``diskette'' icon in the table viewer toolbar to save the
tabular query result as an attachment to the test record. ~Use the
VOTable ``TableData'' format.

\medskip }
\end{minipage}
\\ \cdashline{2-2}


 & Expected Result \\
 & \begin{minipage}[t]{15cm}{\footnotesize

\medskip }
\end{minipage} \\ \cdashline{2-2}

 & Actual Result \\
 & \begin{minipage}[t]{15cm}{\footnotesize

\medskip }
\end{minipage} \\ \cdashline{2-2}

 & Status: \textbf{ Not Executed } \\ \hline

15 & Description \\
 & \begin{minipage}[t]{15cm}
{\footnotesize
Click on the `i'-in-a-circle button in the table viewer. ~Use the
resulting dialog to record the URL for the query job result, and to
download the XML file at that URL and save it as an attachment in the
test record.

\medskip }
\end{minipage}
\\ \cdashline{2-2}


 & Expected Result \\
 & \begin{minipage}[t]{15cm}{\footnotesize

\medskip }
\end{minipage} \\ \cdashline{2-2}

 & Actual Result \\
 & \begin{minipage}[t]{15cm}{\footnotesize

\medskip }
\end{minipage} \\ \cdashline{2-2}

 & Status: \textbf{ Not Executed } \\ \hline

\end{longtable}



\newpage
\appendix
% Make sure lsst-texmf/bin/generateAcronyms.py is in your path
\section{Acronyms used in this document}\label{sec:acronyms}
\input{acronyms.tex}

\newpage

\input{DMTR-301.appendix.tex}

\end{document}
