% generated from JIRA project LVV
% using template at /usr/share/miniconda/envs/docsteady-env/lib/python3.7/site-packages/docsteady/templates/tpr.latex.jinja2.
% using docsteady version 2.2.3
% Please do not edit -- update information in Jira instead
\documentclass[DM,STR,toc]{lsstdoc}
\usepackage{geometry}
\usepackage{longtable,booktabs}
\usepackage{enumitem}
\usepackage{arydshln}
\usepackage{attachfile}
\usepackage{array}
\usepackage{dashrule}

\newcolumntype{L}[1]{>{\raggedright\let\newline\\\arraybackslash\hspace{0pt}}p{#1}}

\input meta.tex

\newcommand{\attachmentsUrl}{https://github.com/\gitorg/\lsstDocType-\lsstDocNum/blob/\gitref/attachments}
\providecommand{\tightlist}{
  \setlength{\itemsep}{0pt}\setlength{\parskip}{0pt}}

\setcounter{tocdepth}{4}

\begin{document}

\def\milestoneName{RSP redeployed on the Interim Data Facility (IDF), ready for DP0.1}
\def\milestoneId{LDM-503-14a}
\def\product{LSP Services}

\setDocCompact{true}

\title{LDM-503-14a: RSP redeployed on the Interim Data Facility (IDF), ready
for DP0.1 Test Plan and Report}
\setDocRef{\lsstDocType-\lsstDocNum}
\date{ 2021-09-28 }
\author{ Gregory Dubois-Felsmann }

% Most recent last
\setDocChangeRecord{
\addtohist{}{2021-03-29}{First draft}{Gregory Dubois-Felsmann}
}

\setDocCurator{Gregory Dubois-Felsmann}
\setDocUpstreamLocation{\url{https://github.com/lsst-dm/\lsstDocType-\lsstDocNum}}
\setDocUpstreamVersion{\vcsrevision}



\setDocAbstract{
This is the test plan and report for
\textbf{ RSP redeployed on the Interim Data Facility (IDF), ready for DP0.1} (LDM-503-14a),
an LSST milestone pertaining to the Data Management Subsystem.\\
This document is based on content automatically extracted from the Jira test database on \docDate.
The most recent change to the document repository was on \vcsDate.
}


\maketitle

\section{Introduction}
\label{sect:intro}


\subsection{Objectives}
\label{sect:objectives}

 Demonstrate that the end-of-FY2020 capabilities of the Rubin Science
Platform have been made available on the Interim Data Facility, and that
DP0.1, based on ingested externally-provided DC2 data, can be supported.
~May be demonstrated with the DC2 DP0.1 dataset itself or with a dataset
of equivalent complexity.\\[2\baselineskip]DP0.1 expectations are as
described in \href{https://rtn-001.lsst.io/}{RTN-001} and
\href{https://rtn-004.lsst.io/}{RTN-004} .



\subsection{System Overview}
\label{sect:systemoverview}



\subsection{Document Overview}
\label{sect:docoverview}

This document was generated from Jira, obtaining the relevant information from the
\href{https://jira.lsstcorp.org/secure/Tests.jspa\#/testPlan/LVV-P79}{LVV-P79}
~Jira Test Plan and related Test Cycles (
\href{https://jira.lsstcorp.org/secure/Tests.jspa\#/testCycle/LVV-C166}{LVV-C166}
).

Section \ref{sect:intro} provides an overview of the test campaign, the system under test (\product{}),
the applicable documentation, and explains how this document is organized.
Section \ref{sect:testplan} provides additional information about the test plan, like for example the configuration
used for this test or related documentation.
Section \ref{sect:personnel} describes the necessary roles and lists the individuals assigned to them.

Section \ref{sect:overview} provides a summary of the test results, including an overview in Table \ref{table:summary},
an overall assessment statement and suggestions for possible improvements.
Section \ref{sect:detailedtestresults} provides detailed results for each step in each test case.

The current status of test plan \href{https://jira.lsstcorp.org/secure/Tests.jspa\#/testPlan/LVV-P79}{LVV-P79} in Jira is \textbf{ Approved }.

\subsection{References}
\label{sect:references}
\renewcommand{\refname}{}
\bibliography{lsst,refs,books,refs_ads,local}


\newpage
\section{Test Plan Details}
\label{sect:testplan}


\subsection{Data Collection}

  Observing is not required for this test campaign.

\subsection{Verification Environment}
\label{sect:hwconf}
  Must be executed in a well-documented controlled state of the IDF.




\subsection{Related Documentation}


No additional documentation provided.


\subsection{PMCS Activity}

Primavera milestones related to the test campaign:
\begin{itemize}
\item LDM-503-14a
\end{itemize}


\newpage
\section{Personnel}
\label{sect:personnel}

The personnel involved in the test campaign is shown in the following table.

{\small
\begin{longtable}{p{3cm}p{3cm}p{3cm}p{6cm}}
\hline
\multicolumn{2}{r}{T. Plan \href{https://jira.lsstcorp.org/secure/Tests.jspa\#/testPlan/LVV-P79}{LVV-P79} owner:} &
\multicolumn{2}{l}{\textbf{ Gregory Dubois-Felsmann } }\\\hline
\multicolumn{2}{r}{T. Cycle \href{https://jira.lsstcorp.org/secure/Tests.jspa\#/testCycle/LVV-C166}{LVV-C166} owner:} &
\multicolumn{2}{l}{\textbf{
Gregory Dubois-Felsmann }
} \\\hline
\textbf{Test Cases} & \textbf{Assigned to} & \textbf{Executed by} & \textbf{Additional Test Personnel} \\ \hline
\href{https://jira.lsstcorp.org/secure/Tests.jspa#/testCase/LVV-T2171}{LVV-T2171}
& {\small Gregory Dubois-Felsmann } & {\small  } &
\begin{minipage}[]{6cm}
\smallskip
{\small Someone with credentials allowing access to the instance of the RSP at
the IDF on which the data are deployed. }
\medskip
\end{minipage}
\\ \hline
\href{https://jira.lsstcorp.org/secure/Tests.jspa#/testCase/LVV-T2172}{LVV-T2172}
& {\small Gregory Dubois-Felsmann } & {\small  } &
\begin{minipage}[]{6cm}
\smallskip
{\small  }
\medskip
\end{minipage}
\\ \hline
\end{longtable}
}

\newpage

\section{Test Campaign Overview}
\label{sect:overview}

\subsection{Summary}
\label{sect:summarytable}

{\small
\begin{longtable}{p{2cm}cp{2.3cm}p{8.6cm}p{2.3cm}}
\toprule
\multicolumn{2}{r}{ T. Plan \href{https://jira.lsstcorp.org/secure/Tests.jspa\#/testPlan/LVV-P79}{LVV-P79}:} &
\multicolumn{2}{p{10.9cm}}{\textbf{ LDM-503-14a: RSP redeployed on the Interim Data Facility (IDF), ready
for DP0.1 }} & Approved \\\hline
\multicolumn{2}{r}{ T. Cycle \href{https://jira.lsstcorp.org/secure/Tests.jspa\#/testCycle/LVV-C166}{LVV-C166}:} &
\multicolumn{2}{p{10.9cm}}{\textbf{ LDM-503-14a: Test RSP capabilities on IDF for DP0.1 readiness }} & Not Executed \\\hline
\textbf{Test Cases} &  \textbf{Ver.} & \textbf{Status} & \textbf{Comment} & \textbf{Issues} \\\toprule
\href{https://jira.lsstcorp.org/secure/Tests.jspa#/testCase/LVV-T2171}{LVV-T2171}
&  1
& Not Executed &
\begin{minipage}[]{9cm}
\smallskip

\medskip
\end{minipage}
&   \\\hline
\href{https://jira.lsstcorp.org/secure/Tests.jspa#/testCase/LVV-T2172}{LVV-T2172}
&  1
& Not Executed &
\begin{minipage}[]{9cm}
\smallskip

\medskip
\end{minipage}
&   \\\hline
\caption{Test Campaign Summary}
\label{table:summary}
\end{longtable}
}

\subsection{Overall Assessment}
\label{sect:overallassessment}

Not yet available.

\subsection{Recommended Improvements}
\label{sect:recommendations}

Not yet available.

\newpage
\section{Detailed Test Results}
\label{sect:detailedtestresults}

\subsection{Test Cycle LVV-C166 }

Open test cycle {\it \href{https://jira.lsstcorp.org/secure/Tests.jspa#/testrun/LVV-C166}{LDM-503-14a: Test RSP capabilities on IDF for DP0.1 readiness}} in Jira.

Test Cycle name: LDM-503-14a: Test RSP capabilities on IDF for DP0.1 readiness\\
Status: Not Executed

This test cycle contains the tests necessary to verify the readiness of
the RSP as redeployed on the IDF to meet the needs of the DP0.1
exercise, essentially repeating tests previously carried out on the NCSA
RSP deployments.

\subsubsection{Software Version/Baseline}
Not provided.

\subsubsection{Configuration}
Not provided.

\subsubsection{Test Cases in LVV-C166 Test Cycle}

\paragraph{ LVV-T2171 - LDM-503-14a: Notebook Aspect access to a DP0.1 dataset in the
IDF-deployed RSP }\mbox{}\\

Version \textbf{1}.
Open  \href{https://jira.lsstcorp.org/secure/Tests.jspa#/testCase/LVV-T2171}{\textit{ LVV-T2171 } }
test case in Jira.

Verify the availability through the Notebook Aspect of the DP0.1 test
dataset or an equivalent, including access to both catalogs and images
via the Butler.\\[2\baselineskip]

\textbf{ Preconditions}:\\
Creation of the DP0.1 dataset or a stand-in, in the form of a Butler
repository accessible from the Notebook Aspect and with associated
catalog data in a TAP service in the same RSP instance at the IDF.

Execution status: {\bf Not Executed }

Final comment:\\


Detailed steps results:

\begin{tabular}{p{2cm}p{14cm}}
\toprule
Step 1 & Step Execution Status: \textbf{ Not Executed } \\ \hline
\end{tabular}
 Description \\
{\footnotesize
In the following step:

\begin{enumerate}
\tightlist
\item
  Note that the URL for the IDF-deployed RSP, provided as test data
  here, is not the NCSA one in the included-by-reference test case
  LVV-T837.
\item
  Please record the options for available releases presented when the
  Notebook Aspect is starting up.
\end{enumerate}

}
\hdashrule[0.5ex]{\textwidth}{1pt}{3mm}
  Test Data \\
 {\footnotesize
https://data.lsst.cloud/

}
\hdashrule[0.5ex]{\textwidth}{1pt}{3mm}
  Expected Result \\
{\footnotesize
Availability of a set of predefined releases is relevant to the
verification of DMS-NB-REQ-0007 (Pre-installed Containerized Software
Releases).

}
\hdashrule[0.5ex]{\textwidth}{1pt}{3mm}
  Actual Result \\
{\footnotesize

}
\begin{tabular}{p{2cm}p{14cm}}
\toprule
Step 2 & Step Execution Status: \textbf{ Not Executed } \\ \hline
\end{tabular}
 Description \\
{\footnotesize
Authenticate to the notebook aspect of the LSST Science Platform
(NB-LSP). ~This is currently at
https://lsst-lsp-stable.ncsa.illinois.edu/nb.

}
\hdashrule[0.5ex]{\textwidth}{1pt}{3mm}
  Expected Result \\
{\footnotesize
Redirection to the spawner page of the NB-LSP allowing selection of the
containerized stack version and machine flavor.

}
\hdashrule[0.5ex]{\textwidth}{1pt}{3mm}
  Actual Result \\
{\footnotesize

}
\begin{tabular}{p{2cm}p{14cm}}
\toprule
Step 3 & Step Execution Status: \textbf{ Not Executed } \\ \hline
\end{tabular}
 Description \\
{\footnotesize
Spawn a container by:\\
1) choosing an appropriate stack version: e.g. the latest weekly.\\
2) choosing an appropriate machine flavor: e.g. medium\\
3) click ``Spawn''

}
\hdashrule[0.5ex]{\textwidth}{1pt}{3mm}
  Expected Result \\
{\footnotesize
Redirection to the JupyterLab environment served from the chosen
container containing the correct stack version.

}
\hdashrule[0.5ex]{\textwidth}{1pt}{3mm}
  Actual Result \\
{\footnotesize

}
\begin{tabular}{p{2cm}p{14cm}}
\toprule
Step 4 & Step Execution Status: \textbf{ Not Executed } \\ \hline
\end{tabular}
 Description \\
{\footnotesize
Record the full URL including scheme of the Notebook Aspect interface in
the browser.

}
\hdashrule[0.5ex]{\textwidth}{1pt}{3mm}
  Expected Result \\
{\footnotesize
The URL should begin with ``https:''. ~This addresses requirement
DMS-NB-REQ-0001 (Secure Protocol).

}
\hdashrule[0.5ex]{\textwidth}{1pt}{3mm}
  Actual Result \\
{\footnotesize

}
\begin{tabular}{p{2cm}p{14cm}}
\toprule
Step 5 & Step Execution Status: \textbf{ Not Executed } \\ \hline
\end{tabular}
 Description \\
{\footnotesize
Use the file browser on the left of the JupyterLab UI to open the ``LSST
Catalog Access Tutorial'' notebook.

}
\hdashrule[0.5ex]{\textwidth}{1pt}{3mm}
  Expected Result \\
{\footnotesize
Execution of this notebook and the one in Step 5 provide test coverage
for the ``notebook interface'' part of the requirement DMS-NB-REQ-0005
(Interactive Python Environment).

}
\hdashrule[0.5ex]{\textwidth}{1pt}{3mm}
  Actual Result \\
{\footnotesize

}
\begin{tabular}{p{2cm}p{14cm}}
\toprule
Step 6 & Step Execution Status: \textbf{ Not Executed } \\ \hline
\end{tabular}
 Description \\
{\footnotesize
Execute the notebook. ~Take note of any errors encountered along the
way.\\[2\baselineskip]Record explicitly that API Aspect TAP queries were
performed.

}
\hdashrule[0.5ex]{\textwidth}{1pt}{3mm}
  Expected Result \\
{\footnotesize
Access to the API Aspect's TAP service provides partial test coverage
for DMS-NB-REQ-0017 ({Access to the API and Portal Aspects).}

}
\hdashrule[0.5ex]{\textwidth}{1pt}{3mm}
  Actual Result \\
{\footnotesize

}
\begin{tabular}{p{2cm}p{14cm}}
\toprule
Step 7 & Step Execution Status: \textbf{ Not Executed } \\ \hline
\end{tabular}
 Description \\
{\footnotesize
Use the file browser on the left of the JupyterLab UI to open the
``Firefly'' notebook.

}
\hdashrule[0.5ex]{\textwidth}{1pt}{3mm}
  Expected Result \\
{\footnotesize

}
\hdashrule[0.5ex]{\textwidth}{1pt}{3mm}
  Actual Result \\
{\footnotesize

}
\begin{tabular}{p{2cm}p{14cm}}
\toprule
Step 8 & Step Execution Status: \textbf{ Not Executed } \\ \hline
\end{tabular}
 Description \\
{\footnotesize
Execute the notebook. ~Take note of any errors encountered along the
way.\\[2\baselineskip]Record explicitly the interaction with the Butler
in this test notebook.\\[2\baselineskip]Record explicitly the use of
afw.display and the observation of images being displayed using the
Portal (Firefly) tools.

}
\hdashrule[0.5ex]{\textwidth}{1pt}{3mm}
  Expected Result \\
{\footnotesize
Successful use of the Butler provides test coverage for DMS-NB-REQ-0009
(Data Access Middleware Availability).\\[2\baselineskip]Availability of
afw.display in the notebook and its use to visualize an image obtained
from the Butler provides test coverage for DMS-NB-REQ-0032 (Image
Visualization).\\[2\baselineskip]Successful use of the Portal for image
visualization provides test coverage for DMS-NB-REQ-0030 (Access to
Portal Visualization API).

}
\hdashrule[0.5ex]{\textwidth}{1pt}{3mm}
  Actual Result \\
{\footnotesize

}
\begin{tabular}{p{2cm}p{14cm}}
\toprule
Step 9 & Step Execution Status: \textbf{ Not Executed } \\ \hline
\end{tabular}
 Description \\
{\footnotesize
Log out of the Notebook Aspect.

}
\hdashrule[0.5ex]{\textwidth}{1pt}{3mm}
  Expected Result \\
{\footnotesize

}
\hdashrule[0.5ex]{\textwidth}{1pt}{3mm}
  Actual Result \\
{\footnotesize

}

\paragraph{ LVV-T2172 - LDM-503-14a: Portal Aspect access to a DP0.1 dataset in the IDF-deployed
RSP }\mbox{}\\

Version \textbf{1}.
Open  \href{https://jira.lsstcorp.org/secure/Tests.jspa#/testCase/LVV-T2172}{\textit{ LVV-T2172 } }
test case in Jira.

Verify the availability through the Portal Aspect to catalog data from
the DP0.1 test dataset or an equivalent, via an RSP TAP service on the
IDF. ~The emphasis will be on an Object-like catalog.\\[2\baselineskip]

\textbf{ Preconditions}:\\
Creation of the DP0.1 dataset or a stand-in, and service of the
associated catalog data and schema in a TAP service in the same RSP
instance at the IDF.

Execution status: {\bf Not Executed }

Final comment:\\


Detailed steps results:

\begin{tabular}{p{2cm}p{14cm}}
\toprule
Step 1 & Step Execution Status: \textbf{ Not Executed } \\ \hline
\end{tabular}
 Description \\
{\footnotesize
Navigate to the Portal Aspect endpoint. ~The stable version should be
used for this test and is currently located at:
https://lsst-lsp-stable.ncsa.illinois.edu/portal/app/ .

}
\hdashrule[0.5ex]{\textwidth}{1pt}{3mm}
  Expected Result \\
{\footnotesize
A credential-entry screen should be displayed.

}
\hdashrule[0.5ex]{\textwidth}{1pt}{3mm}
  Actual Result \\
{\footnotesize

}
\begin{tabular}{p{2cm}p{14cm}}
\toprule
Step 2 & Step Execution Status: \textbf{ Not Executed } \\ \hline
\end{tabular}
 Description \\
{\footnotesize
Enter a valid set of credentials for an LSST user with LSP access on the
instance under test.

}
\hdashrule[0.5ex]{\textwidth}{1pt}{3mm}
  Expected Result \\
{\footnotesize
The Portal Aspect UI should be displayed following authentication.

}
\hdashrule[0.5ex]{\textwidth}{1pt}{3mm}
  Actual Result \\
{\footnotesize

}
\begin{tabular}{p{2cm}p{14cm}}
\toprule
Step 3 & Step Execution Status: \textbf{ Not Executed } \\ \hline
\end{tabular}
 Description \\
{\footnotesize
Navigate to the TAP Search screen

}
\hdashrule[0.5ex]{\textwidth}{1pt}{3mm}
  Expected Result \\
{\footnotesize

}
\hdashrule[0.5ex]{\textwidth}{1pt}{3mm}
  Actual Result \\
{\footnotesize

}
\begin{tabular}{p{2cm}p{14cm}}
\toprule
Step 4 & Step Execution Status: \textbf{ Not Executed } \\ \hline
\end{tabular}
 Description \\
{\footnotesize
Ensure that the TAP service internal to the RSP instance is selected.
(This should be the default choice.)\\[2\baselineskip]Record the list of
available table collections (``schemas'').

}
\hdashrule[0.5ex]{\textwidth}{1pt}{3mm}
  Expected Result \\
{\footnotesize
A list of ``schemas'' available on that service should be displayed,
along with a list of tables in the default schema.

}
\hdashrule[0.5ex]{\textwidth}{1pt}{3mm}
  Actual Result \\
{\footnotesize

}
\begin{tabular}{p{2cm}p{14cm}}
\toprule
Step 5 & Step Execution Status: \textbf{ Not Executed } \\ \hline
\end{tabular}
 Description \\
{\footnotesize
Select the TAP table collection / ``schema'' for the data to be queried
(see test parameter).\\[2\baselineskip]Record the list of available
tables in that schema.

}
\hdashrule[0.5ex]{\textwidth}{1pt}{3mm}
  Test Data \\
 {\footnotesize
{dp01\_dc2\_catalogs}⁠~

}
\hdashrule[0.5ex]{\textwidth}{1pt}{3mm}
  Expected Result \\
{\footnotesize
A list of tables in the selected schema should be
displayed.\\[2\baselineskip]The ability to query any table from the list
of schemas and tables is intended to satisfy DMS-PRTL-REQ-0015 (Generic
Query).

}
\hdashrule[0.5ex]{\textwidth}{1pt}{3mm}
  Actual Result \\
{\footnotesize

}
\begin{tabular}{p{2cm}p{14cm}}
\toprule
Step 6 & Step Execution Status: \textbf{ Not Executed } \\ \hline
\end{tabular}
 Description \\
{\footnotesize
Select the catalog table to be queried (see test parameter).

}
\hdashrule[0.5ex]{\textwidth}{1pt}{3mm}
  Test Data \\
 {\footnotesize
{object}⁠~

}
\hdashrule[0.5ex]{\textwidth}{1pt}{3mm}
  Expected Result \\
{\footnotesize
A search interface for the selected table should be
presented.\\[2\baselineskip]This interface is intended to satisfy
DMS-PRTL-REQ-0016 (Generic Query - Form-Based).

}
\hdashrule[0.5ex]{\textwidth}{1pt}{3mm}
  Actual Result \\
{\footnotesize

}
\begin{tabular}{p{2cm}p{14cm}}
\toprule
Step 7 & Step Execution Status: \textbf{ Not Executed } \\ \hline
\end{tabular}
 Description \\
{\footnotesize
Enter the sky coordinates of the location to be tested (see test
parameter) in the ``Spatial'' query-builder element on the left of the
screen.\\
(Note that the test dataset is likely to be of limited extent on the
sky.) Enter 100 arcseconds as the search radius.

}
\hdashrule[0.5ex]{\textwidth}{1pt}{3mm}
  Test Data \\
 {\footnotesize
{(60,-35)}⁠~

}
\hdashrule[0.5ex]{\textwidth}{1pt}{3mm}
  Expected Result \\
{\footnotesize
The cone-search form query interface tested here is intended to satisfy
DMS-PRTL-REQ-0020 (Positional Query: Position on the Sky) and
DMS-PRTL-REQ-0026 (Positional Query by Region: Cone-Search).

}
\hdashrule[0.5ex]{\textwidth}{1pt}{3mm}
  Actual Result \\
{\footnotesize

}
\begin{tabular}{p{2cm}p{14cm}}
\toprule
Step 8 & Step Execution Status: \textbf{ Not Executed } \\ \hline
\end{tabular}
 Description \\
{\footnotesize
Verify that a list of available columns is displayed on the right of the
search screen. ~Note in the test report whether a subset of the
available columns is highlighted with a check mark, and if so which
columns they are.

}
\hdashrule[0.5ex]{\textwidth}{1pt}{3mm}
  Expected Result \\
{\footnotesize

}
\hdashrule[0.5ex]{\textwidth}{1pt}{3mm}
  Actual Result \\
{\footnotesize

}
\begin{tabular}{p{2cm}p{14cm}}
\toprule
Step 9 & Step Execution Status: \textbf{ Not Executed } \\ \hline
\end{tabular}
 Description \\
{\footnotesize
Execute the search.

}
\hdashrule[0.5ex]{\textwidth}{1pt}{3mm}
  Expected Result \\
{\footnotesize
Possibly after the display of an in-progress indication, a search result
should be displayed in the ``tri-view'' - a coverage image on the upper
left, a default X-Y plot on the upper right, and the tabular query
result on the bottom.\\[2\baselineskip]Note that if the dataset is
simulated, the coverage image may not correspond to the catalog data.
(It is not a requirement of DP0.1 for a coverage image for DESC DC2 to
be created or made available in the Portal Aspect.)\\[2\baselineskip]The
table viewer is intended to satisfy DMS-PRTL-REQ-0049 (Display of
Tabular Data).\\[2\baselineskip]Overplotting of the table on the
coverage image is intended to satisfy DMS-PRTL-REQ-0076 (Image Plot
Overlays) and DMS-PRTL-REQ-0098 (Overlay Catalog of Sources and Objects
on Images). ~Additional aspects of this are tested in subsequent steps.

}
\hdashrule[0.5ex]{\textwidth}{1pt}{3mm}
  Actual Result \\
{\footnotesize

}
\begin{tabular}{p{2cm}p{14cm}}
\toprule
Step 10 & Step Execution Status: \textbf{ Not Executed } \\ \hline
\end{tabular}
 Description \\
{\footnotesize
Verify that the query result covers the expected region of sky, and that
the expected set of columns is included in the query result. ~Record the
number of rows returned by the query.\\[2\baselineskip]Record the
identity displayed for the coverage image.

}
\hdashrule[0.5ex]{\textwidth}{1pt}{3mm}
  Expected Result \\
{\footnotesize

}
\hdashrule[0.5ex]{\textwidth}{1pt}{3mm}
  Actual Result \\
{\footnotesize

}
\begin{tabular}{p{2cm}p{14cm}}
\toprule
Step 11 & Step Execution Status: \textbf{ Not Executed } \\ \hline
\end{tabular}
 Description \\
{\footnotesize
Verify that the set of columns displayed in the table can be modified
using the Table Options dialog available from the ``gears'' button in
the table toolbar.

}
\hdashrule[0.5ex]{\textwidth}{1pt}{3mm}
  Expected Result \\
{\footnotesize
The ability to select a subset of the available columns for viewing is
intended to satisfy DMS-PRTL-REQ-0050 (Column Selection of Tabular
Data).

}
\hdashrule[0.5ex]{\textwidth}{1pt}{3mm}
  Actual Result \\
{\footnotesize

}
\begin{tabular}{p{2cm}p{14cm}}
\toprule
Step 12 & Step Execution Status: \textbf{ Not Executed } \\ \hline
\end{tabular}
 Description \\
{\footnotesize
Verify that the table is displayed in a ``paged'' manner. ~Record the
initial page size. ~Verify that the page size can be changed.

}
\hdashrule[0.5ex]{\textwidth}{1pt}{3mm}
  Expected Result \\
{\footnotesize
The ability to display a paged view of a table is intended to satisfy
DMS-PRTL-REQ-0054 (Paging of Tabular Data).

}
\hdashrule[0.5ex]{\textwidth}{1pt}{3mm}
  Actual Result \\
{\footnotesize

}
\begin{tabular}{p{2cm}p{14cm}}
\toprule
Step 13 & Step Execution Status: \textbf{ Not Executed } \\ \hline
\end{tabular}
 Description \\
{\footnotesize
Verify that the X-Y plot can be modified to display a user-selected pair
of columns, using the plot options dialog available from the ``gears''
button in the plot toolbar.

}
\hdashrule[0.5ex]{\textwidth}{1pt}{3mm}
  Expected Result \\
{\footnotesize
The scatter plot facility is meant to satisfy DMS-PRTL-REQ-0055 (XY
Scatter Plots).

}
\hdashrule[0.5ex]{\textwidth}{1pt}{3mm}
  Actual Result \\
{\footnotesize

}
\begin{tabular}{p{2cm}p{14cm}}
\toprule
Step 14 & Step Execution Status: \textbf{ Not Executed } \\ \hline
\end{tabular}
 Description \\
{\footnotesize
Verify that highlighted rows in the table, points in the X-Y plot, and
marks on the coverage image are connected.

}
\hdashrule[0.5ex]{\textwidth}{1pt}{3mm}
  Expected Result \\
{\footnotesize
The linkages between highlighted points in the three panes of the
tri-view are intended to satisfy DMS-PRTL-REQ-0106 (Linked Tables,
Plots, and Images).

}
\hdashrule[0.5ex]{\textwidth}{1pt}{3mm}
  Actual Result \\
{\footnotesize

}
\begin{tabular}{p{2cm}p{14cm}}
\toprule
Step 15 & Step Execution Status: \textbf{ Not Executed } \\ \hline
\end{tabular}
 Description \\
{\footnotesize
Verify that selections made in the three panes of the tri-view are
reflected in the other panes.

}
\hdashrule[0.5ex]{\textwidth}{1pt}{3mm}
  Expected Result \\
{\footnotesize
The ability to perform linked selections graphically is intended to
satisfy DMS-PRTL-REQ-0107 (Data Selection from a Plot or
Image).\\[2\baselineskip]The ability to perform selections (by
filtering) on a table satisfies DMS-PRTL-REQ-0090 (Simple Filtering of
Tabular Data) and DMS-PRTL-REQ-0092 (Filtering of Tabular Data by
Multiple Columns).

}
\hdashrule[0.5ex]{\textwidth}{1pt}{3mm}
  Actual Result \\
{\footnotesize

}
\begin{tabular}{p{2cm}p{14cm}}
\toprule
Step 16 & Step Execution Status: \textbf{ Not Executed } \\ \hline
\end{tabular}
 Description \\
{\footnotesize
Use the ``diskette'' icon in the table viewer toolbar to save the
tabular query result as an attachment to the test record. ~Use the
VOTable ``TableData'' format, but record all the download format options
presented.

}
\hdashrule[0.5ex]{\textwidth}{1pt}{3mm}
  Expected Result \\
{\footnotesize
This capability partially satisfies (just for the Portal)
DMS-LSP-REQ-0014 (Download Data).\\[2\baselineskip]VOTable and CSV
should be supported for now. For full coverage of DMS-LSP-REQ-0017
(Tabular Data Download File Formats), FITS must also be an
option.\\[2\baselineskip]The ability to save a displayed table is
intended to satisfy DMS-PRTL-REQ-0095 (Saving DIsplayed Tabular Data).

}
\hdashrule[0.5ex]{\textwidth}{1pt}{3mm}
  Actual Result \\
{\footnotesize

}
\begin{tabular}{p{2cm}p{14cm}}
\toprule
Step 17 & Step Execution Status: \textbf{ Not Executed } \\ \hline
\end{tabular}
 Description \\
{\footnotesize
Click on the `i'-in-a-circle button in the table viewer. ~Use the
resulting dialog to record the URL for the query job result, and to
download the XML file at that URL and save it as an attachment in the
test record.

}
\hdashrule[0.5ex]{\textwidth}{1pt}{3mm}
  Expected Result \\
{\footnotesize
This capability provides a basic implementation of the Portal side of
the requirement DMS-LSP-REQ-0010 (Transfer of Portal Data References to
Notebook).

}
\hdashrule[0.5ex]{\textwidth}{1pt}{3mm}
  Actual Result \\
{\footnotesize

}




\newpage
\appendix
% Make sure lsst-texmf/bin/generateAcronyms.py is in your path
\section{Acronyms used in this document}\label{sec:acronyms}
\input{acronyms.tex}

\newpage

\input{DMTR-301.appendix.tex}

\end{document}
